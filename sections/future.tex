Addionaly to the changes explained before, the 2013 Team intend to add some new features to this year competition. We've decided to focus on only four new strategies, mainly because of the time to implement this new changes.


\subsection{Partitioning based on refuges}
	\label{subsec:partition}

The current version of the partitioning algorithm simply divides the map on rectangles based only on the number of Police Forces the map disposes. The new appproach consists of changing the partitioning to make the divisions based on the distribution of refuges on the map. on the new version, one {\it police} agent shall be assigned to each refuge partition. As before, the Police Forces left out of the partitioning shall be used to aid other agents on their actions, by removing blockades, for example.

\subsection{Preventive blockade removal}
	\label{subsec:preventive}

One of the main problems we face on the simulation are the blockades that stop the Ambulances of reaching the dying civilians. To solve this problem, we pretend to use the Police Forces to perform preventive blockade removals, so that Ambulances could easily save their targets. When an Ambulance decides to save a civilian, it will communicate to all agents the location of its target. The closest available Police Force will then move to the target location and, after that, to the closest refuge, removing all the blockages on the way and increasing the chances of survival of the civilian. Since part of the Police Forces are used on the partitioning of the map, only the remaining agents will do the preventive removal.

\subsection{Civilian asking for help}
	\label{subsec:help}
	
The 2011 Team doesn't take into account the cry for help sent by the trapped civilians. By missing this opportunity, we lose the chance of knowing the position of civilians in danger and, as consequence, have a smaller score because of the deceased people. Therefore, this year's team intend to implement, in all the agents, the act of hearing the environment to capture the help call.

\subsection{Rescue payoff}
	\label{subsec:rescuepayoff}
	
Until now, the payoff calculus done by the ambulances are very simple and don't take into account the many variables envolving the rescue of a civilian. We pretend to evolve the calculation of the payoff so it may base on factors such as: distance to target, life of the target, presence of fire near or on the target area, presence of blockades on the way to the target. Using a new payoff, the ambulances shall choose more wisely the civilians to be salved and, therefore, increasing the chances of survival and the score of the team.
