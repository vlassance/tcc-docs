%%
Natural disasters are a source of concern for virtually every country in the world as they usually are difficult to predict and cause huge human and material losses. Altough some disasters may be predicted with a considerable advance, most of them cannot, which leaves a short period of time for any preventive action. Thus, it requires the pre-existence of efficient policies to help handling and responding to such situations, which makes disaster management an important discipline. 

According to \cite{blanchardEtAl2007}, disaster management is comprised of four phases: mitigation, preparedness, response and recovery. In order to promote research on techniques focused on the response phase of disaster management, the RoboCup Rescue Agent Simulation League \cite{kitanoEtAl1999} was founded in 2000. The competition approaches the problem from a multiagent system perspective, in which teams of heterogeneous agents acts on a simulated post-earthquake scenario. This kind of scenario imposes great challenges to the development of teams of agents since the environment is dynamic, the information and communication are limited and unreliable, and the time to make a decision is limited.

The LTI Agent Rescue team is applying for its second participation on the competition. For RoboCup 2013, we focus in some aspects identified as the bottleneck of the team's performance on RoboCup 2011: the map exploration and the coordination strategy.

Last year, the map exploration was performed based on the agents' random walking, however it caused a concentration of the agents in specific regions of the map leaving many parts of the map unexplored, victims undetected and fires unnoticed. Besides, the channel communication characteristics were not explored since the team used a strategy that concentrated all the communication and information on the {\it center} agents.

Therefore, some changes and improvements have being made for the RoboCup 2013 competition. In order to mitigate the map exploration problem, we developed a method of partitioning, used by the Police Forces to divide the map into sectors and reduce the patrolling area of each one of them (see Section \ref{sec:partitioning}). Regarding the coordination approach problem, we changed how the agents exchange information among themselves in order to allow them to communicate directly without requiring the {\it center}'s' intermediation. Thus, it changed the task allocation approach from a hybrid to a completely distributed one. Moreover, some techniques to avoid channel overload, such as the Adaptive Huffman encoding \cite{huffman1952,vitter1987} and interleaved information exchange were implemented (see Section \ref{sec:communication}).

The rest of the document is structured as follows. Section \ref{sec:strategy} describes the distributed strategy employed in the development of each kind of agent, as well as their specific behaviors. Next, the map partitioning technique developed to subdivide the map for Police Forces' surveillance are described in Section \ref{sec:partitioning}. In Section \ref{sec:communication}, we present the communication protocol used to enable the information exchange among the agents. Finally, we present some conclusions in Section \ref{sec:conclusion}.
%%
